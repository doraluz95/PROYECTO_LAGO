% ------------------------------------------------------------------------
% ------------------------------------------------------------------------
%                             Conclusiones
% ------------------------------------------------------------------------
% ------------------------------------------------------------------------

\chapter{Conclusiones}
% Se presenta en forma exacta el aporte del desarrollo den trabajo en concordancia a la justificación presentada.
% Se describe en forma lógica, los resultados del trabajo, dando respuesta a los objetivos o propósitos planteados.
% Basado en los resultados recolectados, incluido el tratamiento estadístico o cualitativo.
% Se muestra en forma concisa los productos y/o resultados y se resaltan las contribuciones del trabajo al contexto local, regional, nacional e internacional, cuando aplique.

% ------------------------------------------------------------------------
En este proyecto se diseñó un sistema de preprocesamiento digital para el registro de aquellos eventos que pueden ser considerados rayos cósmicos, se implementaron circuitos a nivel RTL en una FPGA con herramientas libres, esta versión obedece a una actualización del proyecto LAGO mejorando la frecuencia de muestreo del evento captado de 40~MHz a 50~MHz y conservando el comportamiento en las señales de control.
    
Se almacena la información de lo eventos registrados en una Raspberry Pi3 model B+ comunicada con la FPGA Icoboard por medio de protocolo SPI, además se registró los datos de los periféricos HP03S y GPS que permiten obtener información relacionada con el entorno del WCD (ubicación geográfica, hora, temperatura ambiente y presión atmosférica).


% -------------------------------------------------------------------
