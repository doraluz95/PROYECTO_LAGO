% ------------------------------------------------------------------------
%                                Resumen
% ------------------------------------------------------------------------
\chapter*{Resumen}

\footnotesize{
\begin{description}
  \item[Título:] Implementación de un sistema para el registro de eventos relacionados con
rayos cósmicos secundarios \astfootnote{Trabajo de grado}
  \item[Autor:] Dora Luz Ballesteros Delgado \asttfootnote{Facultad de Ingenierías Físico-Mecánicas. Escuela de Ingenierías Eléctrica, Electrónica y de Telecomunicaciones. Director: Carlos Andrés Angulo Julio, Magíster en Ing. Electrónica.}
  \item[Palabras Clave:] WCD, PMT, Radiación Cherenkov, LAGO, FPGA, EAS.
  \item[Descripción:] 
  Este proyecto tiene como objetivo mejorar el sistema de adquisición de datos usado en la red de colaboración LAGO (\textit{Latin American Giant Observatory}), la cual implementa un sistema digital para la detección de eventos que pueden ser considerados fenómenos debidos a rayos cósmicos.
  Con esta mejora se busca darle continuidad al proyecto LAGO así como independencia del sistema de adquisición respecto a las referencias de FPGA que se encuentran en el mercado ya que en el ciclo completo de diseño se han empleado herramientas de software libre.

El sistema de preprocesamiento de señales se rediseñó para que fuese llevado a cabo por medio de un procesador de propósito específico implementado en la FPGA.
En la FPGA también se implementó un conjunto de bloques jerárquicos tales como: control de voltaje de polarización para los tubos fotomultiplicadores, corrección de línea base y discriminación de datos.
Esto permite aumentar en 10 MHz la frecuencia de muestreo y mejorar la resolución de la señal de entrada hasta 4 veces.

Adicionalmente, el sistema de adquisición integra periféricos para establecer el tiempo de adquisición, la ubicación geográfica, la presión atmosférica y la temperatura ambiente por medio de una Raspberry Pi que toma lecturas de un GPS y de sensores de temperatura y presión.
Así mismo, esta Raspberry Pi se comunica a la FPGA por protocolo SPI para poder generar un reporte y guardar en un archivo la información de las señales registradas.
Este archivo puede ser enviado a otro dispositivo con el fin de analizar los datos de los eventos que pueden ser considerados rayos cósmicos.


%___antiguo

%\item[Descripción:] En este proyecto se busca mejorar el sistema de adquisición de datos usado en la colaboración LAGO y darles continuidad e independencia respecto a las referencias de FPGA que se encuentran temporalmente en el mercado.
%Para esto se implementará un sistema basado en una Raspberry y una FPGA de hardware libre al cual se le ingresará información captada por un detector Cherenkov.
%En la FPGA se implementarán un conjunto de bloques jerárquicos (tales como: discriminación de datos, corrección línea base y control voltaje de polarización para los tubos fotomultiplicadores) que permitan detectar eventos relacionados con rayos cósmicos.
%En la Raspberry se desarrollarán algoritmos que permitan la comunicación con los periféricos para establecer el tiempo de adquisición, la ubicación geográfica, la presión atmosférica y la temperatura ambiente. 
%Así mismo, en este dispositivo se generará y guardará un archivo con la información de las señales registradas.
\end{description}}\normalsize



% ------------------------------------------------------------------------ 