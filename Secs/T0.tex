% ------------------------------------------------------------------------
%                              Introducción
% ------------------------------------------------------------------------
\nnchapter{Introducción}
% ------------------------------------------------------------------------

Los rayos cósmicos son partículas elementales y fragmentos atómicos provenientes del espacio que interactúan con la atmósfera terrestre con un rango de energía desde $10^9$~eV hasta $10^{20}$~eV que viajan casi a la velocidad de la luz. Estos contribuyen a la radiación natural y son uno de los fenómenos más interesantes y desconocidos en la física de partículas y astropartículas ~\citep{Supanitsky2007}.

Cuando un rayo cósmico de alta energía interactúa con los núcleos atómicos de la  atmósfera terrestre desencadena reacciones nucleares que originan nuevas partículas secundarias que se multiplican en un proceso de cascada.~\citep{phdthesis}.

Para la detección de rayos cósmicos se pueden utilizar varios métodos tales como detectores ubicados a nivel del suelo como: WCDs, telescopios de fluorescencia y detectores de centelleo.
Estos métodos tienen como fin caracterizar la energía y dirección de incidencia de las partículas.

LAGO es un observatorio de rayos cósmicos extendido  compuesto detectores Cherenkov de agua (WCD) situados en diferentes paises de América Latina.
El objetivo de dicho proyecto es crear una red de detección de rayos cósmicos con miras a realizar análisis de eventos de clima espacial.
Cada punto en la red de detección consta de WCDs, sistemas para la adquisición de datos, sensores que generan información relevante para el post-procesamiento de las señales y dispositivos de procesamiento y almacenamiento de la información.

Actualmente existen tres WCDs dispuestos dentro de la Universidad Industrial de Santander (UIS) que forman parte del proyecto LAGO.
El arreglo de WCDs tiene forma de triángulo cuasi equilátero con el fin de facilitar el cálculo de los parámetros para la dirección de la EAS mediante la radiación depositada por los rayos cósmicos secundarios al atravesar el tanque detector ~\citep{hernandez2018procedimiento}.

En este  proyecto se  presenta  una  actualización del sistema de registro de eventos  relacionados con rayos cósmicos secundarios para el proyecto LAGO.
El diseño utilizado es una derivación del hardware actual.
Este libro está organizado de la siguiente manera: 
en el capítulo~1 se exponen los objetivos planteados para el proyecto.
El Capítulo~2 contiene los conceptos generales para el desarrollo del proyecto.
En el Capítulo~3 se describe el sistema de adquisición LAGO mientras que en el Capítulo~4 se presenta la implementación del proyecto.
En el Capítulo~5 se muestran los resultados obtenidos y se finaliza con el trabajo futuro y las conclusiones en el Capítulo~6 y el capítulo~7 respectivamente.
 
% ------------------------------------------------------------------------ 