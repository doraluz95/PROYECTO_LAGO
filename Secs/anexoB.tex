% ------------------------------------------------------------------------
%                                Anexo B
% ------------------------------------------------------------------------
\newpage
\anexo{Función \emph{ode45} de MATLAB}\label{anexoB}
% ------------------------------------------------------------------------
\noindent La función $ode45$ está basada en un algoritmo de tipo Runge-Kutta, que se desarrolló a partir del método de Euler mejorado \citep{chapra2003metodos}. La función recibe tres parámetros esenciales: $f(t)$ dentro de un $script$ en el que se define la ecuación diferencial acompañado por un simbolo $@$, el vector de límites de tiempo $\left[ t_0 \quad t_f \right]$ y el vector de condiciones iniciales $y_0$. En otras palabras el prototipo básico para usar \emph{ode45} es el siguiente:
% ------------------------------------------------------------------------
\begin{equation}\label{defode}
   [t,y]=\emph{ode45}(@f(t),\left[ t_0 \quad t_f \right],y_0);
    \end{equation}
% ------------------------------------------------------------------------
En este caso la solución numérica se almacenará en el vector $y$  para cada uno de los instantes de tiempo presentes en el vector $t$.\\

La función $ode45$, resuelve ecuaciones del tipo $\dot{y} = f(t,y)$, por tanto si se desea resolver ecuaciones de orden superior estas deben escribirse como un sistema de ecuaciones diferenciales de primer orden.\\

A manera de ejemplo, se ilustrará la forma de resolver la ecuación diferencial de segundo orden
% ------------------------------------------------------------------------
\begin{equation}\label{segundo}
   \ddot{x}-\mu\left(1-x^2\right)\dot{x}+x=0,
\end{equation}
% ------------------------------------------------------------------------
donde $\mu > 0$ es un parámetro escalar.\\

Por tanto, definiendo
% ------------------------------------------------------------------------
$$y_1 = x; \quad y_2 = \dot{x}$$
% ------------------------------------------------------------------------
la expresión \eqref{segundo} puede ser reescrita como
% ------------------------------------------------------------------------
$$
\dot{y}_2 = \mu \left( 1 - y_1^2 \right) y_2 + y_1
$$
% ------------------------------------------------------------------------
es decir, transformando la ecuación diferencial original de segundo orden y una variable, en una ecuación diferencial equivalente de primer orden y dos variables. Así entonces, es posible construir el vector
% ------------------------------------------------------------------------
$$
y = \left[ {\begin{array}{*{20}c}
   y_1  \\
   y_2  \\
\end{array}} \right]
$$
% ------------------------------------------------------------------------
cuya dinámica viene representada por
% ------------------------------------------------------------------------
$$
\dot{y} = \left[ {\begin{array}{*{20}c}
   f_1 \left(t, y \right)  \\
   f_2 \left(t, y \right) \\
\end{array}} \right]
$$
% ------------------------------------------------------------------------
siendo
% ------------------------------------------------------------------------
$$f_1\left(t, y \right) = y_2; \quad f_2\left(t, y \right) =  \mu \left( 1 - y_1^2 \right) y_2 + y_1$$
% ------------------------------------------------------------------------
De esta manera, evaluar la expresión \eqref{defode} permite obtener una matriz de salida $y$ con filas representando los vectores solución para $y_1$ e $y_2$ como función de $t$.\\
% ------------------------------------------------------------------------

\begin{figure}[h]
\centering
\includegraphics[width=0.55\textwidth]{Figs/flujodiag}
\caption[]{Diagrama de flujo para algoritmo de integración numérica de la función $ode45$ de MATLAB}\label{flujodiag}
\end{figure}
% ------------------------------------------------------------------------

En la Fig. \ref {flujodiag} se ilustra el diagrama de flujo del algoritmo empleado para hallar la solución de una ecuación diferencial mediante integración numérica empleando la función $ode45$ de MATLAB.\\

Inicialmente, se deben asignar los parámetros definidos en la ecuación \eqref{defode}.\\

Posteriormente, un bucle interno hace llamado iterativo a la función $f(t)$ evaluada para valores de tiempo entre $t_0$ y $t_f$ a partir de las condiciones iniciales $y_0$. Para cada ciclo la condición inicial se recalcula siendo la condición final del ciclo anterior. El tiempo se incrementa en un tamaño de paso $\delta t$ de forma adaptativa, si no se especifica lo contrario. Tras alcanzarse el tiempo final $t_f$, el bucle interno termina y entrega como resultado el vector de puntos de la trayectoria solución $y(t)$ al igual que el vector de tiempos $t$. 