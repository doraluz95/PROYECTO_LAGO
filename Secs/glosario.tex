% ------------------------------------------------------------------------
%                                Glosario
% ------------------------------------------------------------------------
\chapter*{Glosario}

\begin{description}
\item[EAS ] \textit{Etensive Air Showers} : Lluvias atmósfericas extendidas o cascada de partículas son partículas secundarias procedentes de espacio iniciadas por la interacción de los núcleos que constituyen la atmósfera con los rayos cósmicos primarios.

\item[FPGA] \textit{Fiel Programmable Gate Array}: Matriz de puertas lógicas programable del inglés field-programmable gate array, es un dispositivo que contiene hardware programable cuya interconexión y funcionalidad puede ser configurada por medio de lenguaje de descripción de hardware (HDL).

\item[LAGO ] \textit{The Latin American Giant Observatory}: observatorio de astropartículas compuesto por detectores Cherenkov en agua ubicados en nueve países de América Latina (Argentina, Bolivia, Brasil, Colombia, Ecuador, Guatemala, México, Perú y Venezuela) y España.
Los objetivos son: estudio del componente de alta energía de las explosiones de rayos gamma a gran altura, analizar y comprender el clima espacial mediante la modulación solar de los rayos cósmicos galácticos y medición de la radiación a nivel del suelo.

\item[PMT] \textit{Photo-Multiplier Tubes} (Tubos fotomultiplicadores):
Detectores de luz extremadamente sensibles diseñados para reaccionar ante una mínima cantidad de luz incidente.
Operan con una fuente de alto voltaje conectada a una red resistiva que polariza sus terminales.
  
\item[Radiación Cherenkov ]
Es la luz emitida por un medio transparente cuando partículas cargadas lo cruzan a una velocidad mayor a la de la luz en ese medio.
La luz se emite en dirección al movimiento de la partícula.

\item[WCD] \textit{Water Cherenkov Detector} (Detector Cherenkov de Agua):
Sistema de detección que consiste en un tanque de agua sellado para evitar el ingreso de luz desde el exterior y recubierto en su interior con un material que permita una alta difusión lumínica. Su funcionamiento se basa en el efecto Cherenkov según el cual cuando una partícula cargada rompe la velocidad de la luz en el medio (agua) genera un cono luminoso de cuya intensidad se puede obtener información del evento.
  
\end{description}
% ------------------------------------------------------------------------